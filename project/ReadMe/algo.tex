\section{Реализация}
\subsection{Подготовка предикторов}
    Для работы с картой используется несколько классов :
    \begin{enumerate}
        \item \textbf{Dot} - точка хранящая координаты по x и y.
        \item \textbf{line} - это отрезки заданные уравнением (имеется функция проверки наличия точки на данном отрезке-\textbf{check()}).
        \item \textbf{point} - движушаяся точка, у которой есть пять векторов направления.
	\end{enumerate}
	
	Также имеется функция  \textbf{cross} позволяющая при помощи простейших формул геометрии искать точки пересечения прямых,постоиных по одному из векторов \textbf{point} и отрезка. Третий параметор это номер вектора по которой нужно строить прямую.
	
	В \textbf{main} созданы точки карты и по ним созданы массивы \textbf{lineS} и \textbf{lineB} ,хранящие нужные отезки контура.
	
	Далее описан поиск расстояний от движушейся точки до отрезков при использовании выше описанных инструментов. При этом сначала ишутся длинна до внешнего конткра,а потом до внутреннего,ведь до последнего расстояние всегда меньше.
	
	Получинные данные записываюся в массив и являются придикаторами.

\subsection{Нейросеть}
    Ответом на вопрос куда нужно двигатся должен быть массив состоящий из 4-х нулей и 1-ой единицы. Так как только по одному направлению точко сможет двигаться дальше.
    
	Для обучения нейросети потребуется верное решение.В данной задаче верным решение может являтся любой массив, который не приведет к столкновению с контуром. Но так как в задаче необходимо найти путь наиболее близкий к внутреннему контуру,то будем выбирать самый левый путь по которому можно ехать(движение осуществляется вправо по "треку".Оно записывается в массив \textbf{Id}.
	
	Перцепртон описан в классе \textbf{perc} , у которого все веса входящие и выходящие записаны в массивах; также имеется функия меняющая виса ; сумматорная функция ,ссумирующая все входящие значения умноженные на их веса.
	
	Активационная функция представленна отдельной функцией.

    В \textbf{mane} получается массив \textbf{distance[]} в качестве параметра вектор значений предикторов одного примера и проводит последовательную
	активацию нейронов по слоям от входного к выходному(значения активационных функций записываются в соответствующие массивы \textbf{column}), после чего возвращает значение активации нейрона в выходном слое в виде массива,наибольшее число которого указывает направление движения.

	Далее реализуется алгоритм обратного распространения ошибки.
	Используется вектор значений предикторов и правильный ответ , а также активационнай функции нейронов и целевоа функции ошибки.

	Сначала считается значение ошибки на выходном слое.	Затем ошибка распространяется на предыдущие слои в обратном порядке.
	Важно что вес каждой связи нужно менять дважды,ведь у каждой связи есть исходный перцептрон и конечный.

	Обучение нейросети происходит до тех пор пока движение становится невозможным, и эти "заезды" повтаряются необходимое число раз для получение нужных весов.

