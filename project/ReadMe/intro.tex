\section{Обзор}
\subsection{Классификация}
	\textbf{Классификация} объектов~--- одна из стандартных задач машинного обучения.
	Её можно описать так: имеется множество объектов, которые каким-то образом разделены на классы.
	Задано конечное множество объектов, для которых известно, к каким классам они относятся.
	Это множество называется \textit{обучающей выборкой}. Классовая принадлежность остальных объектов неизвестна.
	Требуется построить алгоритм, способный классифицировать произвольный объект (то есть указать к какому классу он относится) из исходного множества.

	В машинном обучении задача классификации относится к разделу \textit{обучения с учителем}. Существует также \textit{обучение без учителя},
	когда разделение объектов обучающей выборки на классы не задаётся, и требуется классифицировать объекты только на основе их
	сходства друг с другом. В этом случае принято говорить о \textit{задачах класстеризации}.

	Одним из самых простых типов классификации является \textit{бинарная классификация}, когда различных классов всего два.
	Данный тип служит основой для решения более сложных задач.

\subsection{Методы решения}
	Для решения задач классификации могут использоваться следующие методы:
	\begin{itemize}
		\item Байесовский классификатор;
		\item Решающие деревья;
		\item Логистическая регрессия;
		\item Искусственные нейронные сети.
	\end{itemize}

	\textbf{Байесовский классификатор}~--- тип алгоритмов классификации, основанный  на теореме, утверждающей,
	что если плотности распределения каждого из классов известны, то искомый алгоритм можно выписать в явном аналитическом виде.
	Более того, этот алгоритм оптимален, то есть обладает минимальной вероятностью ошибок.
	На практике плотности распределения классов, как правило, не известны. Их приходится оценивать по обучающей выборке.
	В результате байесовский алгоритм перестаёт быть оптимальным, так как восстановить плотность по выборке можно только с некоторой погрешностью.
	В задаче бинарной классификации звуков восстановление плотности классов является плохо решаемой проблемой.

	\textbf{Решающие деревья}~--- средство поддержки принятия решений, структура которого представляет собой \textit{листья} и \textit{ветки}.
    На ветках дерева записаны атрибуты, от которых зависит целевая функция, в листьях записаны значения целевой функции,
    а в остальных узлах~--- атрибуты, по которым различаются случаи. Цель состоит в том, чтобы создать модель,
    которая предсказывает значение целевой переменной на основе нескольких переменных на входе.

    Одним из основных вопросов в реализации решающих деревьев для задачи классификации является выбор атрибутов,
    по которым будет осуществляться разделение данных на классы.

    \textbf{Логистическая регрессия}~--- метод построения линейной разделяющей поверхности.
    В случае двух классов разделяющей поверхностью является гиперплоскость.
    В задаче бинарной классификации звуков нельзя гарантировать возможность разделения пространства параметров
    одной гиперплоскостью.

    \textbf{Искусственная нейронная сеть}~--- это математическая модель, построенная в некотором смысле по образу
    и подобию сетей нервных клеток живого организма. Для решения задачи классификации может использоваться
    такой тип ИНС, как \textit{многослойный перцептрон Розенблатта}.
    Он представляет собой передающую сеть, состоящую из генераторов сигнала трёх типов: сенсорных элементов,
    ассоциативных элементов и реагирующих элементов. Производящие функции этих элементов зависят от сигналов, возникающих либо где-то внутри передающей сети,
    либо, для внешних элементов, от сигналов, поступающих из внешней среды.
